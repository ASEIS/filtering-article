% 
The validation of ground motion synthetics has received increased attention over the last few years due to advances in physics-based deterministic and hybrid simulation methods. Validation of synthetics is necessary in order to determine whether the available simulation methods are capable of faithfully reproducing the characteristics of ground motions from past earthquakes. Some validation methods use filters to evaluate the quality of the fit between synthetics and data within different frequency bands. This is done, primarily, to weight the contribution of different wavelengths so that low frequencies are given more weight than high frequencies. One particular method of interest is the goodness-of-fit (GOF) criterion introduced by Anderson (2004). In this method, the degree of similarity between two signals is quantified by means of a set of ten (error) metrics which are projected on a 0?10 scoring scale. These metrics are based on ground motion characteristics commonly used in seismology and earthquake engineering. The scores are used to evaluate each given pair of signals in the three components of motion and within different frequency ranges. The scores of each frequency band, component, and metric are then combined to obtain a final GOF score. In this paper we study the sensitivity of the GOF scores, and thus the sensitivity of the validation itself, to the filtering process when different filter parameters are considered. Our initial analysis of results shows that GOF methods are susceptible to the design of filters. The filter?s order, for instance, seems to significantly affect the interpretation of the validation especially for metrics that are time-dependent (e.g., peak ground response). We evaluate the implications of the variability in GOF scores on the 60 randomly generated (white noise) waves. We calculated the best parameters to each filter through Genetic Algorithm. We test best sets of parameters on two case studies of the 2008 Mw 5.4 Chino Hills earthquake, and Broadband Platform, investigate the sensitivity of GOF criteria to the type of filters used to decompose the signals. We analyze the consistency and correlation of the results obtained using various metrics by means of a filtering fitting function. Our work indicates that elliptic infinite impulse response filters lead to more reliable results, over other more commonly used filters; and the filtering parameters are mainly dependent on filtering bandwidth. We provide relationships to select the most accurate parameters of elliptic filter.